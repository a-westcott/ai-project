\documentclass[a4paper,11pt]{article}
\usepackage{amsmath}
\usepackage[margin=2.4cm]{geometry}

\usepackage{fancyhdr}
\pagestyle{fancy}
\lhead{COMP30024 Project Part A Report}
\rhead{Alexander Westcott \& Cameron Chandler}

\author{Alexander Westcott 994344 \& Cameron Chandler 993990}
\title{COMP30024 Artificial Intelligence Project Part A Report}
\date{April 2020}

\begin{document}
\maketitle

\noindent Our implementation of Part A of the project uses code sourced from the
Artificial Intelligence: A Modern Approach GitHub repository: github.com/aimacode/aima-python.

\section*{The Game as a Search Problem}

The game has been formulated as a search problem considering each board configuration as a state. 
Each legal white move in that board configuration is one of the actions that can be taken given that state. 
The goal test simply compares the current board configuration to an empty board. Our path cost is one unit per action. 

\section*{The Algorithm}

A* was chosen as the search algorithm because it is complete in this search space with finite nodes (checks were made for repeated states). 
Empirically, A* was found to be more efficient than other algorithms like iterative deepening. 
Our implementation of A* is not optimal as the heuristic we 
developed is not admissible. 
The chosen heuristic calculates the manhattan distance between all black pieces to all white pieces. 
It sums up the smallest distance for each black piece. $h(node)$ is defined below: \newline

$x_{1,i} =$ x coordinate of $i^{\text{th}}$ black piece $i \in {1,\dots,n}$

$y_{1,i} =$ y coordinate of $i^{\text{th}}$ black piece $i \in {1,\dots,n}$

$x_{2,j} =$ x coordinate of $j^{\text{th}}$ white piece $j \in {1,\dots,m}$

$y_{2,j} =$ y coordinate of $j^{\text{th}}$ white piece $j \in {1,\dots,m}$

Let $S_i$ be the set of Manhattan distances from the $i^{\text{th}}$ black piece to all white pieces.

$S_i = \left\{\left| x_{2,j} - x_{1,i}\right| + \left| y_{2,j} - y_{1,i}\right| :j \in 1,\dots,m\right\}$

$h(node) = \displaystyle\sum_{i=1}^n{\min\left(S_i\right)}$

\section*{Algorithm Complexity}
Let $b$ be the branching factor, $h$ be our heuristic function, $h^*$ be remaining cost to an optimal solution, $\varepsilon = \left(\frac{\left|h-h^* \right|}{h^*}\right)$ be the heuristic error and $d$ be the depth of the least cost solution.
The branching factor affects both the time and space complexity. 
Every node is retained in memory during the search so both the time and space complexity are $O(b^{\varepsilon d})$.
The branching factor depends on the number of white pieces -- increasing with more white pieces -- and can vary slightly depending on the board configuration. 
e.g. how the pieces are stacked, distance to the edge of the board and black pieces in the way of movements.

As our heuristic $(h)$ gets further away from the true value of the remaining cost $(h^*)$, the heuristic error $\varepsilon$ increases, thus incresing the time and space complexity\dots

\end{document}